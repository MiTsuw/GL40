\chapter*{Introduction}
\label{intro_generale}
 \addcontentsline{toc}{chapter}{Introduction}
 \markboth{Introduction}{Introduction}
 
Ce document r�pond � une probl�matique d'optimisation type. La solution logicielle r�pond aux sp�cifications fournies dans le document :
\begin{itemize}
\item 		\textit{cahier\_des\_charges.pdf}
\end{itemize}

Le probl�me d'optimisation trait� est un pro\-bl�me NP-difficile fictif, trait� par m�ta-heuristique. La tr�s grande combinatoire du probl�me et l'�tude des approches d'optimisation de la litt�rature en recherche op�ra\-tionnelle am�ne � consid�rer son traitement par l'usage des m�thodes m�taheuristiques. Ce document pr�sente une solution type pour r�soudre le probl�me. La m�thode est bas�e sur des m�taheu\-ristiques de type recherche locale et algorithme �volu\-tionnaire et m�m�tique.

Les diff�rentes sections sont organis�es pour faire ressortir les principaux composants et �tapes de la conception et mod�lisation de l'algorithme d'opti\-misation et son implantation en C++. Les chapitres pr�sentent successivement les �l�ments suivants :
\begin{itemize}
\item 		\textit{D�finition du probl�me d'optimisation}
\item 		\textit{Principe de r�solution}
\item 		\textit{Structures de donn�es}
\item 		\textit{Op�rateurs de base}
\item 		\textit{Algorithme m�taheuristique}
\item 		\textit{Jeux de tests et �valuation}
\end{itemize}

Une conclusion g�n�rale termine le document.
